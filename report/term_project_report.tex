\documentclass[conference]{IEEEtran}
\IEEEoverridecommandlockouts
% The preceding line is only needed to identify funding in the first footnote. If that is unneeded, please comment it out.
\usepackage{cite}
\usepackage{amsmath,amssymb,amsfonts,amsthm}
\usepackage{algorithmic}
\usepackage{graphicx}
\usepackage{textcomp}
\usepackage{xcolor}
\usepackage{simplebnf}
\usepackage{bussproofs}
\usepackage{hyperref}
\EnableBpAbbreviations
\def\BibTeX{{\rm B\kern-.05em{\sc i\kern-.025em b}\kern-.08em
    T\kern-.1667em\lower.7ex\hbox{E}\kern-.125emX}}
\newcommand\lam[2]{\lambda #1.#2}
\newcommand\unp[3]{\textbf{let }[#1]=#2\textbf{ in }#3}
\newcommand\gradedt[2]{\square_#1 #2}
\newcommand\public{\texttt{Public}}
\newcommand\secret{\texttt{Secret}}
\newcommand\irrele{\texttt{Irrelevant}}

\newtheorem{definition}{Definition}
\newtheorem{theorem}{Theorem}
\newtheorem{lemma}{Lemma}
\begin{document}

\title{\textsc{FlowSTLC}: An Information Flow Control Type System Based On Graded Modality
}

\author{\IEEEauthorblockN{1\textsuperscript{st} Zhige Chen}
\IEEEauthorblockA{\textit{Dept. of Computer Science}\\\textit{and Engineering} \\
\textit{Southern University of}\\\textit{Science and Technology}\\
Shenzhen, China \\
12413315@mail.sustech.edu.cn}
\and
\IEEEauthorblockN{2\textsuperscript{nd} Junqi Huang}
\IEEEauthorblockA{\textit{Dept. of Computer Science}\\\textit{and Engineering} \\
\textit{Southern University of}\\\textit{Science and Technology}\\
Shenzhen, China \\
12212226@mail.sustech.edu.cn}
\and
\IEEEauthorblockN{3\textsuperscript{rd} Zhiyuan Cao}
\IEEEauthorblockA{\textit{Dept. of Computer Science}\\\textit{and Engineering} \\
\textit{Southern University of}\\\textit{Science and Technology}\\
Shenzhen, China \\
12311109@mail.sustech.edu.cn}
}

\maketitle

\begin{abstract}
In this project, we introduce the design and implementation of a simple functional programming language with a type system that enforces secure information flow. Building on the simply-typed lambda calculus (STLC), we extend the type system with a graded modality with a security semiring. Inspired by Graded Modal Dependent Type Theory (Moon, Eades, Orchard), our system statically prevents unauthorized flows of sensitive data into public outputs, ensuring the noninterference property.
\end{abstract}

\begin{IEEEkeywords}
Computer security, information flow, noninterference, security-type systems
\end{IEEEkeywords}

\section{Introduction}

\section{\textsc{FlowSTLC}: The Core Calculus}
\subsection{Syntax}

\subsection{The Security Semiring}

\subsection{Typing}

\subsection{Evaluation}

\subsection{Graded Modality}

\section{Metatheory}
\subsection{Substitution}
\begin{lemma}[Permutation]
	If $\Gamma\vdash t:T$ and $\Delta$ is a permutation of $\Gamma$, then $\Delta\vdash t:T$ and the derivation depth of the latter is the same as the former.
\end{lemma}
The proof for the permutation lemma is trivial since assumptions in contexts are unrelated in a simply-typed context. The permutation lemma enables us to state the two substitution lemmas in a cleaner form, we first prove the well-typedness of substitution through regular assumptions:
\begin{lemma}[Well-typed Substitution] 
	If $\Gamma,x:S\vdash t:T$ and $\Delta\vdash s:S$, then $\Gamma,\Delta\vdash [x\mapsto s]t:T$.
\end{lemma}

\begin{proof}
	By induction on a derivation of the form $\Gamma,x:S\vdash t:T$. For a given derivation, we proceed by cases on the final typing rule used in the proof.
	\begin{itemize}		
		\item\textit{Case} \textsc{T-Var}: $t=z$ with $z:T\in(\Gamma,x:S)$
		
		We need to consider two subcases: 
		\begin{enumerate}
			\item if $z=x$, then $[x\mapsto s]z\to[x\mapsto s]x\to s$. So we need to show $\Gamma,\Delta\vdash s:S$. To get this we apply the \textsc{T-Weak} rule and the permutation lemma to the assumption $\Delta\vdash s:S$ which will give us desired result.
			\item otherwise $[x\mapsto s]z\to z$, and the result is immediate.
		\end{enumerate}	
		
		\item\textit{Case} \textsc{T-Abs}: $t=\lam{y:T_2}{t_1}$, $T=T_2\to T_1$, $\Gamma,x:S,y:T_2\vdash t_1:T_1$
		
		First by the permutation lemma and weakening rule, we have $\Gamma,\Delta,y:T_2,x:S\vdash t_1:T_1$. Again we apply the two rules on the assumption $\Delta\vdash s:S$ to get $\Gamma,\Delta,y:T_2\vdash s:S$. By the induction hypothesis, we have $\Gamma,\Delta,y:T_2\vdash [x\mapsto s]t_1:T_1$, then by \textsc{T-Abs} $\Gamma,\Delta\vdash \lam{y:T_2}{[x\mapsto s]t_1}:T_2\to T_1$, and this is our desired result.
		
		\item\textit{Case} \textsc{T-App}: $t=t_1\; t_2$, $\Gamma,x:S\vdash t_1:T_2\to T_1$, $\Gamma,x:S\vdash t_2:T_2$, $T=T_1$
		
		By the induction hypothesis, we have $\Gamma\vdash [x\mapsto s]t_1:T_2\to T_1$ and $\Gamma\vdash [x\mapsto s]t_2:T_2$. Then by \textsc{T-App}, $\Gamma\vdash [x\mapsto s]t_1\; [x\mapsto s]t_2:T_1$, i.e., $\Gamma\vdash [x\mapsto s](t_1\; t_2):T_1$.
		
		\item\textit{Case} \textsc{T-Weak}: $t=t_1$, $\Gamma'\vdash t_1:T$ where $\Gamma'\subseteq(\Gamma,x:S)$ and $(\Gamma,x:S)-\Gamma'$ does not contain assumptions graded by $\public$.
		
		By the induction hypothesis, we have $\Gamma'\vdash [x\mapsto s]t_1:T$, then apply the \textsc{T-Weak}, we have $\Gamma,x:S\vdash [x\mapsto s]t_1:T$.
		
		\item\textit{Case} \textsc{T-Der}:
		
		\item\textit{Case} \textsc{T-Pro}:
		
		\item\textit{Case} \textsc{T-Let}:
		
		\item\textit{Case} \textsc{T-Approx}:
	\end{itemize}
\end{proof}

\begin{lemma}[Well-typed Graded Substitution] 
	If $\Gamma,x:[S]_\ell\vdash t:T$ and $[\Delta]\vdash s:S$, then $\Gamma,\ell\cdot\Delta\vdash [x\mapsto s]t:T$.
\end{lemma}

\begin{proof}
	
\end{proof}

\subsection{Type Preservation}
\begin{theorem}[Type Preservation]
	
\end{theorem}

\subsection{Progression}

\subsection{Strong Normalization}

\subsection{Noninterference}

\subsection{Type Checking Algorithm}

\section{Implementation and Examples}

\section{Related Work}

\section{Further Work and Conclusion}

\section*{Acknowledgment}

\section*{References}

\appendix

\subsection{Complete Specification of \textsc{FlowSTLC}}
\subsubsection{Syntax}
\begin{center}
	\begin{bnf}
		$t$ : \textsf{Term} ::=
		| $x$ : \textit{variable}
		| $t$ $t$ : \textit{application}
		| $\lam{x}{t}$ : \textit{abstraction}
		| $[t]$ : \textit{packing}
		| $\unp{x}{t\colon T}{t}$ : \textit{unpacking}
		;;
		$T$ : \textsf{Type} ::=
		| $T\to T$ : \textit{function type}
		| $\gradedt{r}{T}$ : \textit{graded modality}
		;;
		$v$ : \textsf{Value} ::=
		| $\lam{x\colon T}{t}$ : \textit{abstraction value}
		;;
		$\Gamma$ : \textsf{Context} ::=
		| $\emptyset$ : \textit{empty context}
        | $\Gamma,x\colon T$ : \textit{assumption}
		| $\Gamma,x\colon[T]_r$ : \textit{graded assumption}
		;;
		$\ell$ : \textsf{Security} ::=
		| \texttt{Secret}
		| \texttt{Public}
		;;
	\end{bnf}
\end{center}

\subsubsection{Security Level Semiring}

\begin{definition}[Security Level Semiring]
The security level semiring is a two-point lattice of security levels $\{\public\sqsubseteq\secret\}$ with 
\begin{itemize}
	\item $0=\secret$
	\item $1=\public$
	\item Addition as the meet: $r+s=r\sqcap s$
	\item Multiplication as the join: $r\cdot s=r\sqcup s$
\end{itemize}
\end{definition}
See \hyperref[app-b]{Appendix B} for a proof that this algebra is a indeed a semiring.

\subsubsection{Auxiliary Definitions}
\begin{definition}[Context Concatenation]
    $$
    \begin{aligned}
        \emptyset+\Gamma&=\Gamma\\
        \Gamma+\emptyset&=\Gamma\\
        (\Gamma,x:T)+\Gamma'&=(\Gamma+\Gamma'),x:T\text{ iff }x\notin dom(\Gamma')\\
        \Gamma+(\Gamma',x:T)&=(\Gamma+\Gamma'),x:T\text{ iff }x\notin dom(\Gamma')\\
        (\Gamma,x:[T]_r)+(\Gamma',x:[T]_s)&=(\Gamma+\Gamma'),x:[T]_{r+s}
    \end{aligned}
    $$
\end{definition}

\begin{definition}[Context Scalar Multiplication]
    $$
    \begin{aligned}
    r\cdot\emptyset&=\emptyset\\
    r\cdot(\Gamma,x:[T]_s)&=(r\cdot\Gamma),x:[T]_{(r\cdot s)}\\
    \end{aligned}
    $$
\end{definition}

\subsubsection{Typing Rules}

\begin{prooftree}
	\AXC{$x:T\in\Gamma$}
	\RightLabel{\textsc{T-Var}}
	\UIC{$\Gamma\vdash x:T$}
\end{prooftree}

\begin{prooftree}
	\AXC{$\Gamma,x:T_1\vdash t:T_2$}
	\RightLabel{\textsc{T-Abs}}
	\UIC{$\Gamma\vdash\lam{x:T_1}{t}:T_1\to T_2$}
\end{prooftree}

\begin{prooftree}
	\AXC{$\Gamma_1\vdash t_1:T_{11}\to T_{12}$}
	\AXC{$\Gamma_2\vdash t_2:T_{11}$}
	\RightLabel{\textsc{T-App}}
	\BIC{$\Gamma_1+\Gamma_2\vdash t_1\; t_2:T_{12}$}
\end{prooftree}

\begin{prooftree}
	\AXC{$\Gamma\vdash t:T$}
	\RightLabel{\textsc{T-Weak}}
	\UIC{$\Gamma,\Gamma'\vdash t:T$}
\end{prooftree}
where $\Gamma'$ denotes a context containing only regular assumptions and assumptions graded by $\public$.

\begin{prooftree}
    \AXC{$\Gamma,x:T_1\vdash t:T_2$}
    \RightLabel{\textsc{T-Der}}
    \UIC{$\Gamma,x:[T_1]_\public\vdash t:T_2$}
\end{prooftree}

\begin{prooftree}
	\AXC{$[\Gamma]\vdash t:T$}
	\RightLabel{\textsc{T-Pro}}
	\UIC{$\ell\cdot[\Gamma]\vdash[t]:\gradedt{\ell}{T}$}
\end{prooftree}

\begin{prooftree}
	\AXC{$\Gamma_1\vdash t_1:\gradedt{\ell}{T_1}$}
	\AXC{$\Gamma_2,x:[T_1]_\ell\vdash t_2:T_2$}
	\RightLabel{\textsc{T-Let}}
	\BIC{$\Gamma_1+\Gamma_2\vdash\unp{x}{t_1}{t_2}:T_2$}
\end{prooftree}

\begin{prooftree}
	\AXC{$\Gamma,x:[T]_{\ell_1},\Gamma'\vdash t:T$}
	\AXC{$\ell_1\sqsubseteq\ell_2$}
	\RightLabel{\textsc{T-Approx}}
	\BIC{$\Gamma,x:[T]_{\ell_2},\Gamma'\vdash t:T$}
\end{prooftree}

\subsubsection{Evaluation Rules}

\begin{prooftree}
	\AXC{$t_1\to t_1'$}
	\RightLabel{\textsc{E-App1}}
	\UIC{$t_1\; t_2\to t_1'\; t_2$}
\end{prooftree}

\begin{prooftree}
	\AXC{$t_2\to t_2'$}
	\RightLabel{\textsc{E-App2}}
	\UIC{$v_1\; t_2\to v_1\; t_2'$}
\end{prooftree}

$$
(\lam{x:T}{t})\; v\to [x\mapsto v]t\quad\textsc{E-AppAbs}
$$

\begin{prooftree}
	\AXC{$t\to t'$}
	\RightLabel{\textsc{E-Pro}}
	\UIC{$[t]\to[t']$}
\end{prooftree}

\begin{prooftree}
	\AXC{$t_1\to t_1'$}
	\RightLabel{\textsc{E-Let-Eval}}
	\UIC{$\unp{x}{t_1}{t_2}\to\unp{x}{t_1'}{t_2}$}
\end{prooftree}

$$
\unp{x}{v}{t}\to [x\mapsto v]t\quad\textsc{E-Let-Unbox}
$$

\subsection{Proofs}
\subsubsection{Security Level Semiring}
\label{app-b}
\begin{proof}\leavevmode
	\begin{itemize}
		\item \textbf{Associativity of addition}: $(a+b)+c=a+(b+c)$
		
		This is trivial since the semiring addition is meet, and meet is associative in a lattice.
		\item \textbf{Commutativity of addition}: $a+b=b+a$
		
		This is also trivial since meet is commutative in a lattice.
		\item \textbf{Additive identity}: $a+0=a$ for all $a$
		
		Since $0=\secret$ and $\public\sqsubseteq\secret$, we have
		\begin{equation}
			\begin{aligned}
				\public\sqcap\secret&=\public\\
				\secret\sqcap\secret&=\secret
			\end{aligned}
		\end{equation}
		\item \textbf{Associativity of multiplication}: $(a\cdot b)\cdot c=a\cdot (b\cdot c)$
		
		This is trivial since the semiring multiplication is join, and join is associative in a lattice.
		
		\item \textbf{Multiplication distributes over addition}: $a\cdot(b+c)=(a\cdot b)+(a\cdot c)$ and $(a+b)\cdot c=(a\cdot c)+(b\cdot c)$
		
		Since the lattice is a chain (and thus distributive), this holds.
		\item \textbf{Multiplicative identity}: $1\cdot a=a\cdot 1=a$ for all $a$
		
		Since $1=\public$ and $\public=\secret$, we have
		\begin{equation}
			\begin{aligned}
				\public\sqcup\public&=\public\\
				\public\sqcup\secret&=\secret\\
				\secret\sqcup\public&=\secret
			\end{aligned}
		\end{equation}
		\item \textbf{Multiplication by 0 annihilates}: $0\cdot a=a\cdot 0=0$ for all $a$
		
		Since $0=\secret$ and $\public\sqsubseteq\secret$, we have
		\begin{equation}
			\begin{aligned}
				\secret\sqcup\public&=\secret\\
				\secret\sqcup\secret&=\secret\\
				\public\sqcup\secret&=\secret
			\end{aligned}
		\end{equation}
	\end{itemize}
	Thus the security level algebra is a semiring.
\end{proof}
\end{document}
